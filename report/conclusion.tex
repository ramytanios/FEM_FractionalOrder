\newpage
\section{Conclusion}
In this project, we have implemented the finite element method for option pricing under Feller--L\'{e}vy models, where the fractional Laplacian operator shows up in the parabolic partial differential equation that governs the price process. The finite element approximation of the latter operator has been established using the sinc quadrature. In the finite element solver, solving a linear system at each step has been done using the conjugate gradient descent iterative method, and that is due to the fact that the finite element matrix approximation matrix of the fractional Laplacian is not known explicitly and hence cannot be inverted. However, the sinc matrix approximation of the fractional Laplacian raised a computational challenge, namely the computationally expensive sum in \eqref{sinc_quad} and future work may be done by parallelizing the latter sum to speed up the program. \\
After highlighting the C++ object-oriented implementation of the finite element method, we showed  for the case of the fractional 1D heat equation an empirical second order convergence of the discretization error in the $L^2$ and $L^{\infty}$ norms that was in accordance with the theoretical rates of convergence presented for the non-fractional case in\cite{schwab}. In addition to that, the linear system showing up in the equation of the sinc matrix has been solved using three different direct and iterative methods, namely the LU-factorization technique, and the conjugate gradient descent with and without preconditioning. We have shown that iterative methods speed up the program especially when the number of mesh nodes is large enough, however the preconditioning seems to be of importance depending on the nature of the problem.