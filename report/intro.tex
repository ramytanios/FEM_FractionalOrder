\section{Introduction}
In mathematical finance 
and especially in option pricing, 
price processes 
are typically modeled as 
solutions 
to stochastic differential equations 
driven either by Brownian motions 
in order to obtain 
continuous-in-time price processes\cite{BS}, 
or by Poisson processes if 
sample paths of 
the price processes are assumed to exhibit
a finite number of jumps\cite{Merton}.

More specifically, in the case of a Brownian motion, 
the stock price is modeled by a large number of independent price changes 
under the assumption of finite variance. 
Dropping the latter assumption, the 
Generalised Central Limit Theorem\cite{GCLT} 
states that the sum of i.i.d.\ random variables with heavy-tailed probability distributions, 
after appropriate shifting and scaling, 
follows a L\'{e}vy-stable distribution. In addition to that, the sequence of partial sums of the i.i.d. random variables then converges to a process that has jumps and indeed not to a Brownian Motion.

Motivated by that, 
we model the stock price by the sum of stochastic increments that follow a 
limiting  L\'{e}vy-stable distribution. 
More specifically, a L\'{e}vy price process $(X_t)_{t\ge0}$ with $X_{t^{-}}{=}\lim\limits_{s\to t, s<t} X_s$ for c\`{a}dl\`{a}g processes, solves 
the L\'{e}vy-driven stochastic differential equation 
\begin{equation} \label{sde}
	\text{d}X_t = b(X_{t^-})\text{d}t + a(X_{t^-})^{1/\alpha}\der L_t^{\alpha}, \ X_0=x,
\end{equation}
where $(L_t^{\alpha})_{t\ge0}$ denotes an $\alpha-$stable L\'{e}vy process, 
$\alpha \in (0,2]$. 

For option pricing under Feller--L\'{e}vy models, we apply in this project deterministic numerical methods targeting the Kolmogorov forward equation in contrast to the widely used and non-deterministic Monte Carlo methods that rely on random numbers generation.



More precisely, we use the \textbf{Feynman--Kac} theorem to state 
the parabolic partial differential equation that governs the option price process, 
where the infinitesimal generator of the Feller process gives rise 
to a fractional power of the Laplace operator 
and thus to a fractional heat equation. 
The fractional Laplacian can then be represented in integral form 
in terms of the non-fractional Laplacian\cite{pazi}. 
We first present the Finite Element semi-discrete form 
(i.e., after discretization in space) 
with a special treatment of the fractional Laplacian, where the integral representation is approximated by a sinc quadrature\cite{Bonito}.
Afterwards, we use the $\theta$-scheme for 
the temporal discretization in order to formulate the fully discrete problem. 
For the implementation part, we present the most important aspects of the object-oriented
C++ implementation of the Finite Element method. We also compare different direct and iterative methods for solving linear systems. Finally, we run a convergence analysis of our FEM discretization 
with regular mesh refinement and for different combinations of the parameters $\theta$ 
and the fractional power of the Laplacian, and present results corresponding to finance applications.
